\documentclass[12pt]{article}

\usepackage[english]{babel}
\usepackage[utf8x]{inputenc}
\usepackage{pdfpages}
\usepackage{lastpage} % Required to determine the last page for the footer
\usepackage{extramarks} % Required for headers and footers
\usepackage{graphicx} % Required to insert images
\usepackage{listings} % Required for insertion of code
\usepackage{courier} % Required for the courier font

% Margins
\topmargin=-0.45in
\evensidemargin=0in
\oddsidemargin=0in
\textwidth=6.5in
\textheight=9.0in
\headsep=0.25in

\linespread{1.1} % Line spacing

\newcommand{\Title}{Requirement Specification} % Assignment title
\newcommand{\Class}{Cos\ 301} % Course/class

\begin{document}

	\vspace{4em}
	
	\begin{center}%
	
	  \LARGE \bf \Title \\[4em]
	  \LARGE {\bf Group 2}\\[1em]
	  \LARGE {\bf Group Members:}\\[2em]
	  \large
	     Nico Taljaard					(10153285) \\[1em]
	     Abraham Daniel Pretorius		(12022404) \\[1em]
	     Mathys Ellis					(12019837) \\[1em]
	     Nbulungo Musetsho				(10176382) \\[1em]
	     Verushka Moodley				(29117454) \\[1em]
	     Eduan Bekker					(12214834) \\[8em]
	     {\bf Version 2.0}
	    
	\end{center}%
	
	\newpage
		{\LARGE \bf Change Log}\\[2em]
		
		\begin{tabbing}
			\hspace*{2.5cm}\=\hspace*{2.5cm}\=\hspace*{8cm}\=\hspace*{3cm} \kill
			10/02/2014 \> Version 1.0 \> Document Created \> Nico Taljaard \\
			15/02/2014 \> Version 1.0 \> System Description \> Mathys Ellis \\
			17/02/2014 \> Version 1.1 \> Edited Formatting \> Nico Taljaard \\
			17/02/2014 \> Version 1.1 \> Technical Specification \> Nbulungo Musetsho \\
			17/02/2014 \> Version 1.1 \> Created Non-functional Requirements \> Eduan Bekker \\
			18/02/2014 \> Version 1.2 \> External Interface Requirements \> Abraham Daniel Pretorius  \\
			18/02/2014 \> Version 1.2 \> Functional Requirement \> Nico Taljaard \\
			18/02/2014 \> Version 1.2 \> Updated Non-functional Requirements \> Eduan Bekker \\
			18/02/2014 \> Version 1.2 \> Updated External Interface Requirements \> Abraham Daniel Pretorius  \\
			18/02/2014 \> Version 1.2 \> Updated System Description \> Mathys Ellis \\
			20/02/2014 \> Version 1.3 \> Fixed Compile Errors \> Nico Taljaard \\
			20/02/2014 \> Version 1.2 \> Updated External Interface Requirements \> Abraham Daniel Pretorius  \\
			21/02/2014 \> Version 1.2 \> Fixed Spelling \> Nico Taljaard \\
			21/02/2014 \> Version 2.0 \> Change formatting \> Nico Taljaard \\
		\end{tabbing}
	
	\newpage
		\tableofcontents	
		
	\newpage
	\section{Introduction}
	
	\vspace{0.2in}

		\subsection{Purpose:}
		\vspace{0.1in}
		The purpose of this document is to communicate the requirements and proposed solution to the client, Mr Jan Kroeze, who requires a computer system for marking purposes. This document will outline the scope of the project and thus serve as a formal agreement and contract between the developers and the client. It will also serve as a reference and eliminate any confusion that may occur in the later stages of development. 
	
	\vspace{0.2in}
	
		\subsection{Document Conventions:}
		\vspace{0.1in}
		\begin{itemize}
			\item Documentation formulation: LaTeX
			\item ERD Crow-Foot notation
			\item UML 2.0
		\end{itemize}
	
	\vspace{0.2in}
	
		\subsection{Project Scope:}
		\vspace{0.1in}
		
		The required computer system should aid the teaching assistants and lecturers in the University of Pretoria to mark assessments of students and the maintenance of marks thereafter. It should also allow students to keep up to date with their marks. 
		
		\vspace{0.5cm}
		
		\begin{flushleft}
		The solution will consist of an application that students, teaching assistants and lecturers will use in the following way:
		\end{flushleft}
			\begin{itemize}
				\item Teaching assistants will receive mark sheets of registered students and will be allowed to assign practical marks to the students via the app. Lecturers will also be granted these permissions.
				\item Students will be able to view a list of their marks via the application
			\end{itemize}
			
		\vspace{0.5cm}
		
		\begin{flushleft}	
		The solution will also consist of a web interface that will allow the following:
		\end{flushleft}
			\begin{itemize}
				\item Lecturers can maintain and integrate marks
				\item Lecturers can request reports with a certain criteria. A graph of some nature or statistics will then be presented 
				\item An audit log will be updated automatically 
			\end{itemize} 
			
		\vspace{0.5cm}
			
		\begin{flushleft}		
		The solution has the following restrictions:
		\end{flushleft}
			\begin{itemize}
				\item Users must be registered- a username and password is required for access to the system
				\item Marking sheets may only be locked/unlocked by lecturers that are assigned to the module. Teaching assistants will be allowed to update marks according to the status of the mark sheet. Students may only view their mark when the mark sheet is locked.
				\item No one may be allowed to edit the audit log. Only lecturers have permission to view the audit log.
			\end{itemize}		

	\vspace{0.2in}
	
		\subsection{References:}
		\vspace{0.1in}
			
	
	\vspace{0.5in}
	
	\section{System Description:}
	\vspace{0.2in}
	
	The goal of the system is to provide the client with a secure, scalable and remotely accessible marking and mark management system. The system is comprised of four different facets which are listed below. It is intended to replace the current marking and mark management system employed by the client, which is currently paper and spreadsheet based.
	
	\vspace{0.5cm}
	\textbf{Student mark retrieval facet:}
	\vspace{0.1in}
	The goal of this facet is to provide the "Students" with a secure and private  means to a read-only view of their marks, for the markable items of a specified course on the system. The means by which they will view the markable items will be in the form of a website interface and android application.
	
	\vspace{0.5cm}
	\textbf{Marker marking tool facet:}
	\vspace{0.1in}
	The goal of this facet is to provide "Markers", assigned to a particular course, with a mobile tool to allow then to be able to add and update the mark of a specified student on the mark list for a particular markable item of the particular course. The tool will allow "Markers" access to the mark list from any location where an internet connection is present. The tool will be in the forms of a website interface and android application.   	
	
	\vspace{0.5cm}
	\textbf{Lecturer mark management facet:}
	\vspace{0.1in}
	The goal of this facet is to provide "Lectures", assigned to a particular course, with a means to manage the marks of each student assigned to the course. Where the term manage comprises of adding, modifying and removing markable items, mark lists and the individual marks of students. Further the facet also has the goal of providing a means to report on mark related data on different levels of granularity of a particular course.  
	
	\vspace{0.5cm}
	\textbf{Audit trail facet:}
	\vspace{0.1in}
	The goal of this facet is to give the system the ability to track all critical actions that occur on the system independent of any user interference and also provide a read-only view of such trails for authorised users of the system. Where critical actions comprise of adding, updating and removing any data on the database as well as login and logout actions.  
	
	\vspace{0.5in}
	
	\section{Functional Requirements:}
	\vspace{0.2in}
		
		\subsection{System features:}
		\vspace{0.15in}
		
		\hspace{0.2in}\textbf{Subject API}
		\newline\textbf{(Priority: High, Requirement: FRQ1)}
		\begin{itemize}
			\item A subject must be creatable.
			\item Lectures must be added to subject by users with correct authorization.
			\item Teaching assistants and tutor able to be added by lectures and be assigned to a practical slot.
			\item Student information for each assigned practical should be pulled from an existing database so that
					teaching assistants and tutor can access it to be marked.
		\end{itemize}
		
		\vspace{0.15in}
		
		\textbf{Lecturer API}
		\textbf{(Priority: High, Requirement: FRQ2)}
		\begin{itemize}
			\item A task must be creatable for the following:
					\begin{itemize}
						\item Practicals
						\item Assignments
						\item Class tests
						\item Tests
					\end{itemize}
			\item Tasks should also contain the following information:
					\begin{itemize}
						\item Starting time to open access.
						\item End time if a specified time is required, else manual locking is required by lecture.
						\item A rubric must be added to show the mark allocation.
					\end{itemize}
			\item Adjust mark weight allocations.
			\item Assign security roles to teaching assistants and tutors.
			\item Move students to a different teaching assistant or tutor.
			\item Use the reporting API.
		\end{itemize}
		
		\vspace{0.15in}
		
		\textbf{Auditing API}
		\textbf{(Priority: High, Requirement: FRQ3)}
		\begin{itemize}
			\item Log file for following activities:
					\begin{itemize}
					 	\item Marks added by whom and when.
					 	\item Marks changed or removed by who, when and what is the reason.
					\end{itemize}
			\item Lectures alone should be able to change marks of his own subjects.
			\item Lectures view own subjects audit logs.
			\item Head of Department alone should assign a user that can view the entire change log.
			\item No edits to the audit log allowed.
		\end{itemize}
		
		\vspace{0.15in}
		
		\textbf{Marking API}
		\textbf{(Priority: High, Requirement: FRQ4)}
		\begin{itemize}
			\item Accessible through mobile application for tutor.
			\item Display all available assessment option for teaching assistants or tutor to mark.
			\item Within assessment display all students registered for current session with a search option.
			\item Search should be available by student number, surname or name. Displaying results containing all three results.
			\item In the selected student the marking rubric should be displayed with type able fields.
			\item Marks should be submitted to the database directly after mark has been finalized for a student.
		\end{itemize}
		
		\vspace{0.15in}
		
		\textbf{Reports API}
		\textbf{(Priority: Medium-High, Requirement: FRQ5)}
		\begin{itemize}
			\item Marks can be exported in a .CSV file containing the selected marks for each student.
			\item All report should be based on a select set of marks.
			\item Numeric statistics can be exported about marks.
			\item Graphical reports should be exported to .pdf file.
		\end{itemize}
		
		\vspace{0.2in}
		
		\textbf{Student API}
		\textbf{(Priority: Medium, Requirement: FRQ6)}
		\begin{itemize}
			\item Accessible through mobile application or web interface.
			\item Landing page displays all subjects of current student.
			\item Subject marks should be viewable for separate assessments as well as a progress mark.
		\end{itemize}
		
	\vspace{0.2in}
	
		\subsection{Use Cases:}
		\vspace{0.1in}
			%\begin{figure}[htbp] %how to add a picture
			%	\centering
			%	\includegraphics[width=\textwidth]{./Project_Use_Case_Diagram}
			%\end{figure}
		
	\vspace{0.2in}
	
		\subsection{Entity Relationship Diagrams:}
		\vspace{0.2in}
		

		\vspace{0.1in}

			\begin{center}
			\textbf{Key Based Diagram}
			\end{center}
		
		\vspace{0.3in}
		
		
		\begin{center}
			\textbf{Fully Attributed Diagram}
			\end{center}
		
		\vspace{0.3in}
		
	\vspace{0.1in}
	
		\subsection{Data Dictionary:}
		\vspace{0.1in}
		
	\vspace{0.5in}
	
	\section{External Interface Requirements:}
	\vspace{0.2in}
	
	The system has many internal and external communication 
	sub-systems which has to interconnect. 
	\\
	The system has to interface with the following external mediums:
	\begin{enumerate}
	\item Internal API for Android systems. Android API version 15 (Ice Cream Sandwich). 
	\item   External API for data transfer between interconnecting systems. 
			This API uses SOAP transfer the data.
	\item	All API's should be able to represent their data from SOAP to:
				\begin{itemize}
					\item XML for interfacing with web servers.
					\item Output marks to .CSV format.					
				\end{itemize} 
	\end{enumerate}
	
	\vspace{0.5in}
	
	\section{Technical Requirements/Non-Functional}
	\vspace{0.2in}
	
		\subsection{Non-function requirements:}% Eduan Bekker
		\vspace{0.1in}
		
		\begin{itemize}
			\item Authentication
				\begin{itemize}
					\item All users first have to login
				\end{itemize}
			\item Audit-ability
				\begin{itemize}
					\item All actions by all users will be added in the audit trails
					\item Audit trails will include:
					\begin{itemize}
						\item Login/logout of users
						\item Modifications of data
					\end{itemize}
				\end{itemize}
			\item Scalability
				\begin{itemize}
					\item Should be able to handle multiple running practicals simultaneously
					\item Performance should not be dependent on the amount of active users
				\end{itemize}
			\item Availability
				\textbf{There will be a 3 levels of privileges:students, markers and lecturers}
				\begin{itemize}
					\item Students will only be able to view marks
					\item Uploading of practical marks by markers should only be available when the practical starts until a time specified by the lecturers
					\item Editing marks will only be available for lecturers
					\item Audit trails can only be viewed by lecturers
					\item Audit trails cannot be modified by anyone
				\end{itemize}
		\end{itemize}
				
		\subsection{Technical specification:} % Mbulungo Musetsho
		\vspace{0.1in}
		
		\begin{itemize}
			\item The system must support the following platforms: 
				\begin{itemize}
					\item Android (API 15)
					\item Web browser (with HTML 5 compatibility)
				\end{itemize}
				
			\item The System must strictly operate over HTTPS
			\item The following technologies and languages must be used for implementation purposes:
				\begin{itemize}
					\item Python with Django (server-side programming)
					\item Java ( Android Module Development)
					\item MySQL (Database)
				\end{itemize}
				\item SOAP interface must be utilized for this system
				\item LDAP (account management)
		\end{itemize}
	
	\vspace{0.5in}
	
	\newpage
	\section{Requirement Matrices:}
	\vspace{0.2in}
	
	
	
	\vspace{0.5in}
	
	\section{Open Issues:} % Eduan Bekker
	\vspace{0.2in}
	
	\begin{itemize}
		\item Theft/loss of mobile devices
		\item User errors like typing errors
	\end{itemize}
	
	
	\vspace{0.5in}
	
	\section{Glossary:}
	\vspace{0.2in}
		
	
	\vspace{0.5in}
		

\end{document}

\documentclass[12pt]{article}

\usepackage[english]{babel}
\usepackage[utf8x]{inputenc}
\usepackage{pdfpages}
\usepackage{lastpage} % Required to determine the last page for the footer
\usepackage{extramarks} % Required for headers and footers
\usepackage{graphicx} % Required to insert images
\usepackage{listings} % Required for insertion of code
\usepackage{courier} % Required for the courier font

% Margins
\topmargin=-0.45in
\evensidemargin=0in
\oddsidemargin=0in
\textwidth=6.5in
\textheight=9.0in
\headsep=0.25in

\linespread{1.1} % Line spacing

\newcommand{\Title}{Requirement Specification} % Assignment title
\newcommand{\Class}{Cos\ 301} % Course/class

\begin{document}

	\vspace{4em}
	
	\begin{center}%
	
	  \LARGE \bf \Title \\[4em]
	  \LARGE {\bf Group 2}\\[1em]
	  \LARGE {\bf Group Members:}\\[2em]
	  \large
	     Nico Taljaard					(10153285) \\[1em]
	     Abraham Daniel Pretorius		(12022404) \\[1em]
	     Mathys Ellis					(12019837) \\[1em]
	     Mbulungo Musetsho				(10176382) \\[1em]
	     Verushka Moodley				(29117454) \\[1em]
	     Eduan Bekker					(12214834) \\[8em]
	     Johan Esterhuyse				(10043285) \\[1em]
	     {\bf Version 3.0}
	    
	\end{center}%
	
	\newpage
		{\LARGE \bf Change Log}\\[2em]
		
		\begin{tabbing}
			\hspace*{2.5cm}\=\hspace*{2.5cm}\=\hspace*{8cm}\=\hspace*{3cm} \kill
			10/02/2014 \> Version 1.0 \> Document Created \> Nico Taljaard \\
			15/02/2014 \> Version 1.0 \> System Description \> Mathys Ellis \\
			17/02/2014 \> Version 1.1 \> Edited Formatting \> Nico Taljaard \\
			17/02/2014 \> Version 1.1 \> Technical Specification \> Mbulungo Musetsho \\
			17/02/2014 \> Version 1.1 \> Created Non-functional Requirements \> Eduan Bekker \\
			18/02/2014 \> Version 1.2 \> External Interface Requirements \> Abraham Daniel Pretorius  \\
			18/02/2014 \> Version 1.2 \> Functional Requirement \> Nico Taljaard \\
			18/02/2014 \> Version 1.2 \> Updated Non-functional Requirements \> Eduan Bekker \\
			18/02/2014 \> Version 1.2 \> Updated External Interface Requirements \> Abraham Daniel Pretorius  \\
			18/02/2014 \> Version 1.2 \> Updated System Description \> Mathys Ellis \\
			20/02/2014 \> Version 1.3 \> Introduction \> Verushka Moodley \\
			20/02/2014 \> Version 1.3 \> Fixed Compile Errors \> Nico Taljaard \\
			20/02/2014 \> Version 1.2 \> Updated External Interface Requirements \> Abraham Daniel Pretorius  \\
			21/02/2014 \> Version 1.2 \> Fixed Spelling \> Nico Taljaard \\
			21/02/2014 \> Version 2.0 \> Change formatting \> Nico Taljaard \\
			23/02/2014 \> Version 2.0 \> Checked spelling and grammar \> Mathys Ellis \\
			24/02/2014 \> Version 3.0 \> Changed to new layout specification \> Nico Taljaard \\
			24/02/2014 \> Version 3.0 \> Required Functionality \> Nico Taljaard \\	
			24/02/2014 \> Version 3.0 \> Quality requirements \> Eduan Bekker \\
			24/02/2014 \> Version 3.0 \> Integration Requirements: Protocols, API specifications  \> Abraham Daniel Pretorius \\	
			26/02/2014 \> Version 3.0 \> Integration Requirements: Integration channel to be used:, \\
			Protocols, API specifications, \\ 
			Quality requirements for integration,
			  \> Abraham Daniel Pretorius \\
			27/02/2014 \> Version 3.0 \> Added use case prioritization and added possible glossary terms with out definitions  \> Mathys Ellis \\
		\end{tabbing}

	\newpage
	\tableofcontents
	
	\newpage
	\section{Introduction}
	
		\vspace{0.2in}
	
				 This document is the software requirements specification for a computerised marking system to be used by the University of Pretoria. It begins by addressing the purpose and vision of the project then continues with a general discussion on the background of the project. Thereafter, the architectural requirements are identified with a main focus on the access channel requirements, quality requirements, integration requirements, and architectural constraints. The function requirements follow by discussing the application functionality required by the users of the system.  In this section of the document, many diagrams are used to depict the flow of communication and interaction between the users and the system. It will also illustrate the processes that the system will need to complete and the states that the system will undergo.  The next topic will communicate the open issues on some of the requirements, and the document concludes with a glossary of possibly unfamiliar terms used in this document.
		
		 
	
	\section{Vision}
	
		\vspace{0.2in}
		
			 		 The purpose of this document is to communicate the requirements and proposed solution to the client, Mr Jan Kroeze, who requires a computer system for marking purposes. This document will outline the scope of the project and thus serve as a formal agreement and contract between the developers and the client. It will also serve as a reference and eliminate any confusion that may occur in the later stages of development.
				
		
	
	\section{Background}
	
		\vspace{0.2in}
		 The University of Pretoria currently has a manual paper based marking system which has many flaws (losing marks is one of the major issues). Our solution aims to assist them by providing a computerised system. The required computer system should aid the teaching assistants and lecturers in the university in recording the marks of assessments for students, and the maintenance of marks thereafter. It should also allow students to keep up to date with their marks. 
			The solution will consist of an application that students, teaching assistants and lecturers will use in the following way:
			\begin{itemize}
						\item Teaching assistants will receive mark sheets of registered students and will be allowed to assign practical marks to the students via the app. Lecturers will also be granted these permissions.
						\item Students will be able to view a list of their marks via the application
			\end{itemize}			
			The solution will also consist of a web interface that will allow the following:
			\begin{itemize}
						\item Lecturers can maintain and integrate marks
						\item Lecturers can request reports with a certain criteria. A graph of some nature or statistics will then be presented 
						\item An audit log will be updated automatically 
			\end{itemize}
			The solution has the following restrictions:
			\begin{itemize}
						\item Users must be registered- a username and password is required for access to the system
						\item Marking sheets may only be locked/unlocked by lecturers that are assigned to the module. Teaching assistants will be allowed to update marks according to the status of the mark sheet. Students may only view their mark when the mark sheet is locked. 
						\item No one may be allowed to edit the audit log. Only lecturers have permission to view the audit log. 
			\end{itemize}
			
		
	
	\section{Architecture requirements}
	
		\vspace{0.2in}
		
		\subsection{Access channel requirements}
		%Please check this is repeted in Architecture constraint. Redundancy?
			\vspace{0.2in}
			\begin{itemize}
						\item The system must support the following platforms: 
							\begin{itemize}
								\item Android (API 15)
								\item Web browser (with HTML 5 compatibility)
							\end{itemize}
							
						\item The system must strictly operate over HTTPS
						\item The following technologies and languages must be used for implementation purposes:
							\begin{itemize}
								\item Python with Django (server-side programming)
								\item Java ( Android Module Development)
								\item MySQL (Database)
							\end{itemize}
							\item SOAP interface must be utilized for this system
							\item LDAP (account management)
			\end{itemize}
			
		
		\subsection{Quality requirements}
		
			\vspace{0.2in}
			\begin{itemize}
					\item Authentication
						\begin{itemize}
							\item All users have to login before they are able to access the system
						\end{itemize}
					\item Audit-ability
						\begin{itemize}
							\item All the actions of all the users will be added to the audit trail log
							\item The events consider as audit trails will include:
							\begin{itemize}
								\item The login and logout of users
								\item The modifications of any data
							\end{itemize}
						\end{itemize}
					\item Scalability
						\begin{itemize}
							\item The system should be able to handle multiple running practicals simultaneously
							\item The performance of the system should not be dependent on the number of active users
						\end{itemize}
					\item Availability
						\textbf{There will be 3 levels of privileges: students, markers and lecturers}
						\begin{itemize}
							\item Students will only be able to view marks
							\item Uploading of practical marks by markers should only be available when the practical starts until a time specified by the lecturers
							\item Editing marks will only be available for lecturers
							\item Audit trails can only be viewed by lecturers
							\item Audit trails cannot be modified by anyone
						\end{itemize}
				\end{itemize}
			
		
		\subsection{Integration requirements}
		
			\vspace{0.2in}


		\begin{flushleft}
				\textbf{Integration channel to be used:}
				\newline\textbf{(Priority: High, Requirement: ARQIRQ1)} 
		\end{flushleft}
			
			\vspace{0.05in} 
			\begin{itemize} 
				\item SOAP (Simple Object Access Protocol) will be used as an interface\
				 between different mediums and platforms to interact with the system.
			\end{itemize} 
			
			\vspace{0.15in}




		\begin{flushleft}
				\textbf{Protocols}
				\newline\textbf{(Priority: High, Requirement: ARQIRQ2)} 
		\end{flushleft}
			
			\vspace{0.05in} 
			
			\begin{enumerate}
				\item SOAP (Simple Object Access Protocol) will be used to transfer 
				structured information over the network (namely the Internet). 
			\end{enumerate}	
			
			\vspace{0.15in}
			
		
\begin{flushleft}
				\textbf{API specifications}
				\newline\textbf{(Priority: High, Requirement: ARQIRQ3)}
\end{flushleft}		
		\vspace{0.05in} 
		
		The system has to interface with the following external mediums:
	\begin{enumerate}
		\item Internal API for Android systems. Android API version 15 (Ice Cream Sandwich). 
		\item External API for data transfer between interconnecting systems. This API uses SOAP 				transfer the data.
	\end{enumerate}
	\vspace{0.05in} 
	
	All API's should represent their data from SOAP to:
	\begin{enumerate}
					\item XML for interfacing with web servers using WSDL.
					\item Output marks to .CSV format.					
	\end{enumerate}
		
		\vspace{0.15in}
			

	
\begin{flushleft}
				\textbf{Quality requirements for integration}
				\newline\textbf{(Priority: High, Requirement: ARQIRQ4)}
\end{flushleft}		

		
		\vspace{0.05in}
			
	\begin{enumerate}
	  \item Performance:		
		\begin{itemize} 
		 \item People viewing the marks should not need to wait extensively, //
		 		however the system can only be as fast as the connection to it.//
		 		Therefor the system cannot be expected to be faster than the connection\\
		 		that is being used.
		 \item Documentation drawn up from the marks may take between  \\
			   10 - 30 seconds to be compiled.
		 \item Markers should be able to update marks within 10 seconds of each other.
		\end{itemize}	

		\item Scalability:		
			\begin{itemize}
			\item The system will be designed with design patterns. These are used so that\\
			the it is easy to upgrade the different aspects independently. \\   
			\end{itemize}

	\item Reliability:		
			\begin{itemize}
			\item A software issue should not arise, but if it does the application \\
			must handle the problem and send an appropriate error message to the user. \\
			If data has failed to send the application must reattempt to upload the data \\
			for a specified amount of time (1 minute) or until such a time that the user \\
			aborts data transfer.
			
			\end{itemize}
			
			
	\item Security:		
			\begin{itemize}
			\item No user, who is unauthorized, may gain access to the system.
			\item Authorized users can only access the aspects of the system as \\
			allocated to them by the administrator.
			\item Security questions, as well as email addresses will be linked to each\\
				of the users. This will be used for password recovery.
			\end{itemize}
			
			
	\item Auditability:		
			\begin{itemize}
			\item Each time marks are changed it will be recorded in an audit trail which not\\
			even the system administrator would be able to edit.
			\end{itemize}
			
\end{enumerate}			
		
	
			\vspace{0.2in}			
			
			
				
		\subsection{Architecture constraints}
	
		\vspace{0.2in}
		
		\begin{flushleft}
				\textbf{Technologies which MUST be used}
				\newline\textbf{(Priority: High, Requirement: ARQ1)}
		\end{flushleft}
			
			\vspace{0.05in}
		\begin{itemize}
					\item The system must support the following platforms: 
						\begin{itemize}
							\item Android (API 15)
							\item Web browser (with HTML 5 compatibility)
						\end{itemize}
						
					\item The system must strictly operate over HTTPS
					\item The following technologies and languages must be used for implementation purposes:
						\begin{itemize}
							\item Python with Django (server-side programming)
							\item Java ( Android Module Development)
							\item MySQL (Database)
						\end{itemize}
						\item SOAP interface must be utilized for this system
						\item LDAP (account management)
		\end{itemize}
		
		\vspace{0.05in}
		
	
	\section{Functional requirements}
	
		\vspace{0.2in}
		
		\subsection{Introduction}
			\begin{flushleft}
				\vspace{0.2in}
	
				The goal of the system is to provide the client with a secure, scalable and remotely accessible marking and mark management system. The system is comprised of four different facets which are listed below. It is intended to replace the current marking and mark management system employed by the client, which is currently paper and spreadsheet based.
				
				\vspace{0.5cm}
				\textbf{Student mark retrieval facet:}
				\vspace{0.1in}
				\\
				The goal of this facet is to provide the "Students" with a secure and private means to a read-only view of their marks, for the markable items of a specified course on the system. The means by which they will view the markable items will be in the form of a website interface and android application.
				
				\vspace{0.5cm}
				\textbf{Marker marking tool facet:}
				\vspace{0.1in}
				\\
				The goal of this facet is to provide "Markers", assigned to a particular course, with a mobile tool to allow then to be able to add and update the mark of a specified student on the mark list for a particular markable item of the particular course. The tool will allow "Markers" access to the mark list from any location where an internet connection is present. The tool will be in the forms of a website interface and android application.
				
				\vspace{0.5cm}
				\textbf{Lecturer mark management facet:}
				\vspace{0.1in}
				\\
				The goal of this facet is to provide "Lectures", assigned to a particular course, with a means to manage the marks of each student assigned to the course and mark sturcture of the course. Where the term manage comprises of adding, modifying and removing markable items, mark lists and the individual marks of students. Further the facet also has the goal of providing a means to report on mark related data on different levels of granularity of a particular course.
				
				\vspace{0.5cm}
				\textbf{Audit trail facet:}
				\vspace{0.1in}
				\\
				The goal of this facet is to give the system the ability to track all critical actions that occur on the system independent of any user interference and also provide a read-only view of such trails for authorised users of the system. Where critical actions comprise of adding, updating and removing any data on the database as well as login and logout actions.
	
				\vspace{0.2in}
			\end{flushleft}
			
		\subsection{Scope and Limitations/Exclusions}
		
			\vspace{0.2in}
			
			
		
		\subsection{Required functionality}
		
			\vspace{0.2in}
			
			\begin{flushleft}
				\textbf{Course API}
				\newline\textbf{(Priority: High, Requirement: FRQ1)}
			\end{flushleft}
			
			\vspace{0.05in}
			
				\begin{itemize}
					\item A course must be creatable.
					\item Lectures must be added to course by users with correct authorization.
					\item Lectures must be able to add Teaching assistants and Tutors to the course, aswell as be able to assign them to a practical time slot.
					\item Student information for each student that is assigned to a practical time slot of a course should be pulled from an existing database so that teaching assistants and tutors of the course can find the student to be marked.
				\end{itemize}
				
			\vspace{0.15in}
				
			\begin{flushleft}
				\textbf{Lecturer API}
				\textbf{(Priority: High, Requirement: FRQ2)}
			\end{flushleft}
			
				\begin{itemize}
					\item A tasks must be creatable for the following:
							\begin{itemize}
								\item Practicals
								\item Assignments
								\item Class tests
								\item Tests
							\end{itemize}
					\item Tasks should also contain the following information:
							\begin{itemize}
								\item Starting time to open access.
								\item End time if a specified time is required, else manual locking is required by lecture.
								\item A rubric must be added to show the mark allocation.
							\end{itemize}
					\item Adjust mark weight allocations.
					\item Lectures alone should be able to change marks of his own subjects.
					\item Assign security roles to teaching assistants and tutors.
					\item Move students to a different teaching assistant or tutor.
					\item Use the reporting API.
				\end{itemize}
				
			\vspace{0.15in}
			
			\begin{flushleft}	
				\textbf{Auditing API}
				\textbf{(Priority: High, Requirement: FRQ3)}
			\end{flushleft}
			
				\begin{itemize}
					\item Log file for following activities:
							\begin{itemize}
							 	\item Marks added by whom and when.
							 	\item Marks changed or removed by who, when and what is the reason.
							 	\item Login and logout activites
							\end{itemize}
		
					\item Lectures must be able view only the audit logs of the course(s) they are assigned to.
					\item Head of Department alone should assign a user that can view the entire change log.
					\item No edits to the audit log allowed.
				\end{itemize}
				
			\vspace{0.15in}
				
			\begin{flushleft}
				\textbf{Marking API}
				\textbf{(Priority: High, Requirement: FRQ4)}
			\end{flushleft}
			
				\begin{itemize}
					\item Accessible through mobile application for teaching assistants or tutors.
					\item Display all available mark lists for teaching assistants or tutors to mark.
					\item Within mark list display all students registered for current session with a search option.
					\item Search filters should be available for student number, surname or name. The displayed results should contain all the results that match the filters.
					\item For the selected student the marking rubric should be displayed with type able fields.
					\item Marks should be submitted to the database directly after mark has been finalized for a student.
				\end{itemize}
				
			\vspace{0.15in}
			
			\begin{flushleft}
				\textbf{Reports API}
				\textbf{(Priority: Medium-High, Requirement: FRQ5)}
			\end{flushleft}
			
				\begin{itemize}
					\item Marks can be exported in a .csv file containing the selected marks for each student.
					\item All reports should be based on a select set of marks.
					\item Numeric statistics can be exported about marks.
					\item Graphical reports should be exported to .pdf file.
				\end{itemize}
				
			\vspace{0.2in}
			
			\begin{flushleft}
				\textbf{Student API}
				\textbf{(Priority: Medium, Requirement: FRQ6)}
			\end{flushleft}
			
				\begin{itemize}
					\item Accessible through mobile application or web interface.
					\item Landing page displays all subjects of current student.
					\item Subject marks should be viewable for separate assessments as well as a progress mark of a particular course.
				\end{itemize}
				
			\vspace{0.2in}
		
		\subsection{Use case prioritization}
		
			\vspace{0.2in}
			
			\begin{itemize}
				\item \textbf{Critical:}
					\begin{itemize}
						\item Log in [1.1; 2.1; 3.1; 4.1]
						\item Create assessment [1.10]
						\item Set rubric [1.12]
						\item Set constraints [1.13]
						\item Set markers [1.14]
						\item Select assessment [1.11; 2.3; 3.5]
						\item Mark assessment [2.5]
						\item Submit marks [2.6]
						\item Select course [1.2; 2.2; 3.2; 4.2]						
						\item Select student [1.6; 2.4]	
						\item View marks [3.3]
						\item View overall marks [1.3]					
						\item Export marks [1.4]
						\item View entire audit trail [4.4]
					\end{itemize}
				\item \textbf{Important:}
					\begin{itemize}
						\item View student marks [1.5]
						\item View students overall marks [1.8]						
						\item Update assessment [1.9]
						\item Update rubric
						\item Update markers
						\item Update constraints
						\item View assessment mark [3.4]
					\end{itemize}
				\item \textbf{Nice-to-Have:}
					\begin{itemize}
						\item Export entire audit trail [4.4]
						\item Select user [4.5]
						\item View individual audit trail [4.3]
						\item Export user audit trail
					\end{itemize}
			\end{itemize}
			
			
		
		\subsection{Use case/Services contracts}
		
			\vspace{0.2in}
			
			\begin{itemize}
				\item \textbf{Pre-Conditions:}
				\item \textbf{Post-Conditions:}
				\item \textbf{Request and Results Data Structures:}
			\end{itemize}
		
		\subsection{Process specifications}
		
			\vspace{0.2in}
			
			
		
		\subsection{Domain Objects}
		
			\vspace{0.2in}
			
			
	
	\section{Open Issues}
	
		\vspace{0.2in}
		
		
	\newpage
	\section{Glossary}
	
		\vspace{0.2in}
		\begin{itemize}
			\item Student -
			\item Marker -
			\item Lecturer -
			
			\item Markable item -
			\item Mark list -			
			\item Course -
			
			\item LDAP -
			\item API -			
			\item HTTPS -
			\item HTML 5 -
			\item PDF -
			\item XML -
			\item CSV -			
			\item SOAP - Simple Object Access Protocol	
			\item WSDL - Web Service Definition Language
			\item Android -
			\item Django -
			\item Python -
			\item Java -
			\item MySQL -
		\end{itemize}
			
	
	
\end{document}

\documentclass[12pt]{article}

\usepackage[english]{babel}
\usepackage[utf8x]{inputenc}
\usepackage{pdfpages}
\usepackage{lastpage} % Required to determine the last page for the footer
\usepackage{extramarks} % Required for headers and footers
\usepackage{graphicx} % Required to insert images
\usepackage{listings} % Required for insertion of code
\usepackage{courier} % Required for the courier font

% Margins
\topmargin=-0.45in
\evensidemargin=0in
\oddsidemargin=0in
\textwidth=6.5in
\textheight=9.0in
\headsep=0.25in

\linespread{1.1} % Line spacing

\newcommand{\Title}{Requirement Specification} % Assignment title
\newcommand{\Class}{Cos\ 301} % Course/class

\begin{document}

	\vspace{4em}
	
	\begin{center}%
	
	  \LARGE \bf \Title \\[4em]
	  \LARGE {\bf Group 2}\\[1em]
	  \LARGE {\bf Group Members:}\\[2em]
	  \large
	     Nico Taljaard					(10153285) \\[1em]
	     Abraham Daniel Pretorius		(12022404) \\[1em]
	     Mathys Ellis					(12019837) \\[1em]
	     Nbulungo Musetsho				(10176382) \\[1em]
	     Verushka Moodley				(29117454) \\[1em]
	     Eduan Bekker					(12214834) \\[8em]
	     {\bf Version 3.0}
	    
	\end{center}%
	
	\newpage
		{\LARGE \bf Change Log}\\[2em]
		
		\begin{tabbing}
			\hspace*{2.5cm}\=\hspace*{2.5cm}\=\hspace*{8cm}\=\hspace*{3cm} \kill
			10/02/2014 \> Version 1.0 \> Document Created \> Nico Taljaard \\
			15/02/2014 \> Version 1.0 \> System Description \> Mathys Ellis \\
			17/02/2014 \> Version 1.1 \> Edited Formatting \> Nico Taljaard \\
			17/02/2014 \> Version 1.1 \> Technical Specification \> Nbulungo Musetsho \\
			17/02/2014 \> Version 1.1 \> Created Non-functional Requirements \> Eduan Bekker \\
			18/02/2014 \> Version 1.2 \> External Interface Requirements \> Abraham Daniel Pretorius  \\
			18/02/2014 \> Version 1.2 \> Functional Requirement \> Nico Taljaard \\
			18/02/2014 \> Version 1.2 \> Updated Non-functional Requirements \> Eduan Bekker \\
			18/02/2014 \> Version 1.2 \> Updated External Interface Requirements \> Abraham Daniel Pretorius  \\
			18/02/2014 \> Version 1.2 \> Updated System Description \> Mathys Ellis \\
			20/02/2014 \> Version 1.3 \> Introduction \> Verushka Moodley \\
			20/02/2014 \> Version 1.3 \> Fixed Compile Errors \> Nico Taljaard \\
			20/02/2014 \> Version 1.2 \> Updated External Interface Requirements \> Abraham Daniel Pretorius  \\
			21/02/2014 \> Version 1.2 \> Fixed Spelling \> Nico Taljaard \\
			21/02/2014 \> Version 2.0 \> Change formatting \> Nico Taljaard \\
			23/02/2014 \> Version 2.0 \> Checked spelling and grammar \> Mathys Ellis \\
			24/02/2014 \> Version 3.0 \> Changed to new layout specification \> Nico Taljaard \\
			24/02/2014 \> Version 3.0 \> Required Functionality \> Nico Taljaard \\	
			24/02/2014 \> Version 3.0 \> Quality requirements \> Eduan Bekker \\
			24/02/2014 \> Version 3.0 \> Integration Requirements: Protocols, API specifications  \> Abraham Daniel Pretorius \\	
		\end{tabbing}

	\newpage
	\tableofcontents
	
	\newpage
	\section{Introduction}
	
		\vspace{0.2in}
	
	\section{Vision}
	
		\vspace{0.2in}
		
		
	
	\section{Background}
	
		\vspace{0.2in}
		
		
	
	\section{Architecture requirements}
	
		\vspace{0.2in}
		
		\subsection{Access channel requirements}
		
			\vspace{0.2in}
			
			
		
		\subsection{Quality requirements}
		
			\vspace{0.2in}
			\begin{itemize}
					\item Authentication
						\begin{itemize}
							\item All users have to login before they are able to access the system
						\end{itemize}
					\item Audit-ability
						\begin{itemize}
							\item All the actions of all the users will be added to the audit trail log
							\item The events consider as audit trails will include:
							\begin{itemize}
								\item The login and logout of users
								\item The modifications of any data
							\end{itemize}
						\end{itemize}
					\item Scalability
						\begin{itemize}
							\item The system should be able to handle multiple running practicals simultaneously
							\item The performance of the system should not be dependent on the number of active users
						\end{itemize}
					\item Availability
						\textbf{There will be 3 levels of privileges: students, markers and lecturers}
						\begin{itemize}
							\item Students will only be able to view marks
							\item Uploading of practical marks by markers should only be available when the practical starts until a time specified by the lecturers
							\item Editing marks will only be available for lecturers
							\item Audit trails can only be viewed by lecturers
							\item Audit trails cannot be modified by anyone
						\end{itemize}
				\end{itemize}
			
		
		\subsection{Integration requirements}
		
			\vspace{0.2in}

		\begin{flushleft}
				\textbf{Protocols}
				\newline\textbf{(Priority: High, Requirement: ARQIRQ3)} 
		\end{flushleft}
			
			\vspace{0.05in} 
			
			\begin{enumerate}
				\item SOAP (Simple Object Access Protocol) will be used to transfer 
				structured information over the network (namely the Internet). 
			\end{enumerate}	
			
			\vspace{0.15in}
			
		
\begin{flushleft}
				\textbf{API specifications}
				\newline\textbf{(Priority: High, Requirement: ARQIRQ3)}
\end{flushleft}		
		\vspace{0.05in} 
		
		The system has to interface with the following external mediums:
	\begin{enumerate}
		\item Internal API for Android systems. Android API version 15 (Ice Cream Sandwich). 
		\item External API for data transfer between interconnecting systems. This API uses SOAP 				transfer the data.
	\end{enumerate}
	\vspace{0.05in} 
	
	All API's should represent their data from SOAP to:
	\begin{enumerate}
					\item XML for interfacing with web servers using WSDL.
					\item Output marks to .CSV format.					
	\end{enumerate}
		
		\vspace{0.15in}
			

	
			\vspace{0.2in}			
			
			
		
		\subsection{Architecture constraints}
	
		\vspace{0.2in}
		
		\begin{flushleft}
				\textbf{Technologies which MUST be used}
				\newline\textbf{(Priority: High, Requirement: ARQ1)}
		\end{flushleft}
			
			\vspace{0.05in}
		
	
	\section{Functional requirements}
	
		\vspace{0.2in}
		
		\subsection{Introduction}
			\begin{flushleft}
				\vspace{0.2in}
	
				The goal of the system is to provide the client with a secure, scalable and remotely accessible marking and mark management system. The system is comprised of four different facets which are listed below. It is intended to replace the current marking and mark management system employed by the client, which is currently paper and spreadsheet based.
				
				\vspace{0.5cm}
				\textbf{Student mark retrieval facet:}
				\vspace{0.1in}
				\\
				The goal of this facet is to provide the "Students" with a secure and private means to a read-only view of their marks, for the markable items of a specified course on the system. The means by which they will view the markable items will be in the form of a website interface and android application.
				
				\vspace{0.5cm}
				\textbf{Marker marking tool facet:}
				\vspace{0.1in}
				\\
				The goal of this facet is to provide "Markers", assigned to a particular course, with a mobile tool to allow then to be able to add and update the mark of a specified student on the mark list for a particular markable item of the particular course. The tool will allow "Markers" access to the mark list from any location where an internet connection is present. The tool will be in the forms of a website interface and android application.
				
				\vspace{0.5cm}
				\textbf{Lecturer mark management facet:}
				\vspace{0.1in}
				\\
				The goal of this facet is to provide "Lectures", assigned to a particular course, with a means to manage the marks of each student assigned to the course and mark sturcture of the course. Where the term manage comprises of adding, modifying and removing markable items, mark lists and the individual marks of students. Further the facet also has the goal of providing a means to report on mark related data on different levels of granularity of a particular course.
				
				\vspace{0.5cm}
				\textbf{Audit trail facet:}
				\vspace{0.1in}
				\\
				The goal of this facet is to give the system the ability to track all critical actions that occur on the system independent of any user interference and also provide a read-only view of such trails for authorised users of the system. Where critical actions comprise of adding, updating and removing any data on the database as well as login and logout actions.
	
				\vspace{0.2in}
			\end{flushleft}
			
		\subsection{Scope and Limitations/Exclusions}
		
			\vspace{0.2in}
			
			
		
		\subsection{Required functionality}
		
			\vspace{0.2in}
			
			\begin{flushleft}
				\textbf{Course API}
				\newline\textbf{(Priority: High, Requirement: FRQ1)}
			\end{flushleft}
			
			\vspace{0.05in}
			
				\begin{itemize}
					\item A course must be creatable.
					\item Lectures must be added to course by users with correct authorization.
					\item Lectures must be able to add Teaching assistants and Tutors to the course, aswell as be able to assign them to a practical time slot.
					\item Student information for each student that is assigned to a practical time slot of a course should be pulled from an existing database so that teaching assistants and tutors of the course can find the student to be marked.
				\end{itemize}
				
			\vspace{0.15in}
				
			\begin{flushleft}
				\textbf{Lecturer API}
				\textbf{(Priority: High, Requirement: FRQ2)}
			\end{flushleft}
			
				\begin{itemize}
					\item A tasks must be creatable for the following:
							\begin{itemize}
								\item Practicals
								\item Assignments
								\item Class tests
								\item Tests
							\end{itemize}
					\item Tasks should also contain the following information:
							\begin{itemize}
								\item Starting time to open access.
								\item End time if a specified time is required, else manual locking is required by lecture.
								\item A rubric must be added to show the mark allocation.
							\end{itemize}
					\item Adjust mark weight allocations.
					\item Lectures alone should be able to change marks of his own subjects.
					\item Assign security roles to teaching assistants and tutors.
					\item Move students to a different teaching assistant or tutor.
					\item Use the reporting API.
				\end{itemize}
				
			\vspace{0.15in}
			
			\begin{flushleft}	
				\textbf{Auditing API}
				\textbf{(Priority: High, Requirement: FRQ3)}
			\end{flushleft}
			
				\begin{itemize}
					\item Log file for following activities:
							\begin{itemize}
							 	\item Marks added by whom and when.
							 	\item Marks changed or removed by who, when and what is the reason.
							 	\item Login and logout activites
							\end{itemize}
		
					\item Lectures must be able view only the audit logs of the course(s) they are assigned to.
					\item Head of Department alone should assign a user that can view the entire change log.
					\item No edits to the audit log allowed.
				\end{itemize}
				
			\vspace{0.15in}
				
			\begin{flushleft}
				\textbf{Marking API}
				\textbf{(Priority: High, Requirement: FRQ4)}
			\end{flushleft}
			
				\begin{itemize}
					\item Accessible through mobile application for teaching assistants or tutors.
					\item Display all available mark lists for teaching assistants or tutors to mark.
					\item Within mark list display all students registered for current session with a search option.
					\item Search filters should be available for student number, surname or name. The displayed results should contain all the results that match the filters.
					\item For the selected student the marking rubric should be displayed with type able fields.
					\item Marks should be submitted to the database directly after mark has been finalized for a student.
				\end{itemize}
				
			\vspace{0.15in}
			
			\begin{flushleft}
				\textbf{Reports API}
				\textbf{(Priority: Medium-High, Requirement: FRQ5)}
			\end{flushleft}
			
				\begin{itemize}
					\item Marks can be exported in a .csv file containing the selected marks for each student.
					\item All reports should be based on a select set of marks.
					\item Numeric statistics can be exported about marks.
					\item Graphical reports should be exported to .pdf file.
				\end{itemize}
				
			\vspace{0.2in}
			
			\begin{flushleft}
				\textbf{Student API}
				\textbf{(Priority: Medium, Requirement: FRQ6)}
			\end{flushleft}
			
				\begin{itemize}
					\item Accessible through mobile application or web interface.
					\item Landing page displays all subjects of current student.
					\item Subject marks should be viewable for separate assessments as well as a progress mark of a particular course.
				\end{itemize}
				
			\vspace{0.2in}
		
		\subsection{Use case prioritization}
		
			\vspace{0.2in}
			
			
		
		\subsection{Use case/Services contracts}
		
			\vspace{0.2in}
			
			
		
		\subsection{Process specifications}
		
			\vspace{0.2in}
			
			
		
		\subsection{Domain Objects}
		
			\vspace{0.2in}
			
			
	
	\section{Open Issues}
	
		\vspace{0.2in}
		
		
	
	\section{Glossary}
	
		\vspace{0.2in}
	SOAP - Simple Object Access Protocol	
	WSDL - Web Service Definition Language	
	
	
\end{document}
